\renewcommand\thechapter{}
\makeatletter
    \setlength\@fptop{0\p@}
\makeatother
\setcounter{chapter}{0}

\newcounter{cha}
\newcommand\chap[1]{%
	\stepcounter{cha}%
	\chapter*{A\arabic{cha} #1}%
	\addcontentsline{toc}{section}{A\arabic{cha} #1}%
}
\numberwithin{figure}{cha}
\numberwithin{table}{cha}

\begin{refsection}[OwnPublication.bib]
% KEIN Textverweis in Literaturverzeichnis voranstellen
\renewbibmacro*{begentry}{}
\nocite{*}
\printbibliography[heading=bibintoc,title={List of own publications}]
\end{refsection}


\if@twoside\cleardoubleoddpage\else\clearpage\fi%
\pagenumbering{Roman}%
%!!!Dieser Wert muss angepasst werden, abhängig davon, wie viele römische Seitenzahlen für Vorwort, Abkürzungen, Inhaltsverzeichnis, etc. im ersten Teil erforderlich sind.
\setcounter{page}{10}

%\chapter{Appendices}
\phantomsection
\addcontentsline{toc}{chapter}{Appendix}
\chapter*{Appendix}
\chaptermark{Appendix}

\renewcommand\thechapter{A\arabic{cha}}
\renewcommand{\thetable}{A\arabic{cha}.\arabic{table}}
\renewcommand{\thefigure}{A\arabic{cha}.\arabic{figure}}
\makeatletter
    \setlength\@fptop{0\p@}
\makeatother
\setcounter{chapter}{0}


{\let\clearpage\relax \chap{Lorem ipsum}}

	Lorem ipsum.

\chap{List of simulation parameters}

    \begin{table}[htb]
    \centering
        \caption{List of simulation parameters.}
        \label{tab:SimulationParameters}
        \splabel{tab:SimulationParameters}%Dieser Verweis wird genutzt, um im Haupttext auf das Anhangs-Kapitel zu verweisen.
        \begin{tabular}{p{6cm} C{6cm}}
            Lorem ipsum & Lorem ipsum \\ \midrule
            Lorem ipsum & Lorem ipsum \\
        \end{tabular}
    \end{table}
